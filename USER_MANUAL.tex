\documentclass[11pt,a4paper]{article}
\usepackage[utf8]{inputenc}
\usepackage[T1]{fontenc}
\usepackage[german,english]{babel}
\usepackage{geometry}
\geometry{a4paper, margin=2.5cm, headheight=15pt}
\usepackage{hyperref}
\usepackage{graphicx}
\usepackage{enumitem}
\usepackage{tocloft}
\usepackage{fancyhdr}
\usepackage{listings}
\usepackage{xcolor}
\usepackage{pmboxdraw}

\hypersetup{
    colorlinks=true,
    linkcolor=blue,
    urlcolor=blue,
    pdftitle={Survey Workbench v2.0 - User Manual},
    pdfauthor={}
}

\pagestyle{fancy}
\fancyhf{}
\fancyhead[L]{Survey Workbench v2.0}
\fancyhead[R]{\thepage}
\renewcommand{\headrulewidth}{0.4pt}

\title{\textbf{Survey Workbench v2.0}\\[0.5em]\Large User Manual}
\author{}
\date{February 2026}

\begin{document}

\maketitle
\thispagestyle{empty}

\vfill
\begin{center}
\textit{Version 2.0 -- Internal Use Only}
\end{center}
\newpage

\tableofcontents
\newpage

\section{Overview}

The Survey Workbench is a comprehensive tool for managing participant folders and extracting survey data from questionnaires. Version 2.0 introduces dynamic questionnaire support, allowing unlimited questionnaire types per participant.

\subsection{Key Features}
\begin{itemize}
    \item Dynamic questionnaire configuration (unlimited types)
    \item Automated participant folder generation
    \item Data extraction from completed questionnaires to Excel
    \item Configuration save/load system
    \item Individual questionnaire customization
    \item Memory-optimized processing
\end{itemize}

\subsection{System Requirements}
\begin{itemize}
    \item Windows 10/11
    \item Microsoft Excel (for data extraction)
    \item Sufficient disk space for participant folders
\end{itemize}

\section{Getting Started}

When you first launch Survey Workbench v2.0, you'll see the main window with two primary sections:

\begin{enumerate}
    \item \textbf{Generate Participant Folder} -- Creates organized folders with questionnaires
    \item \textbf{Extract Data} -- Reads completed questionnaires and appends to masterfile (CSV, XLS, or XLSX)
\end{enumerate}

\section{Generate Participant Folder}

This section allows you to create standardized participant folders with multiple questionnaire types.

\subsection{Step-by-Step Guide}

\subsubsection{Step 1: Enter Participant ID}
Enter a unique identifier for the participant (e.g., ``WaS3\_001'').

\begin{itemize}
    \item This ID will be used in folder and file names
    \item Use consistent naming across all participants
    \item The field cannot be empty
\end{itemize}

\subsubsection{Step 2: Select Target Folder}
\begin{enumerate}
    \item Click \textbf{Search} next to ``Target folder PATH''
    \item Choose where participant folders will be created
    \item Example: \texttt{C:\textbackslash Studies\textbackslash WaS3\textbackslash Participants}
\end{enumerate}

\subsubsection{Step 3: Configure Questionnaire Types}
\begin{enumerate}
    \item Enter the number of different questionnaire types needed
    \item Click \textbf{Confirm} to create configuration rows
\end{enumerate}

A scrollable area will appear with configuration fields for each questionnaire type.

\subsubsection{Step 4: Configure Each Questionnaire}

For each questionnaire type, you'll configure three parameters:

\paragraph{Survey Name:}
\begin{itemize}
    \item Enter the questionnaire identifier (e.g., ``demographics'', ``nasa1'', ``comments'')
    \item This name will be part of the output filename
    \item Use descriptive, consistent names
\end{itemize}

\paragraph{Template PATH:}
\begin{itemize}
    \item Click \textbf{Search} to select the PDF template file
    \item Example: \texttt{C:\textbackslash Templates\textbackslash WaS3\_demographics.pdf}
    \item Template must be a valid PDF file
\end{itemize}

\paragraph{Number of copies:}
\begin{itemize}
    \item Enter how many copies of this questionnaire to create
    \item Default is 1
    \item Use multiple copies for repeated measures (e.g., 8 copies for NASA-TLX)
\end{itemize}

\subsubsection{Step 5: Generate Folder}

Click \textbf{Generate Participant Folder}. The system will:

\begin{enumerate}
    \item Create a folder named with the Participant ID
    \item Copy all questionnaires with standardized naming
    \item Open the folder automatically
\end{enumerate}

\subsection{File Naming Convention}

\begin{itemize}
    \item Single copy: \texttt{\{ParticipantID\}\_\{SurveyName\}.pdf}
    \item Multiple copies: \texttt{\{ParticipantID\}\_\{SurveyName\}1.pdf}, \texttt{\{ParticipantID\}\_\{SurveyName\}2.pdf}, etc.
\end{itemize}

\subsection{Example Output}

\begin{verbatim}
WaS3_001/
├── WaS3_001_demographics.pdf
├── WaS3_001_nasa1.pdf
├── WaS3_001_nasa2.pdf
├── WaS3_001_nasa3.pdf
├── WaS3_001_nasa4.pdf
├── WaS3_001_nasa5.pdf
├── WaS3_001_nasa6.pdf
├── WaS3_001_nasa7.pdf
├── WaS3_001_nasa8.pdf
└── WaS3_001_comments.pdf
\end{verbatim}

\section{Extract Data}

This section reads completed questionnaire data and appends it to a masterfile. The system automatically detects the format (CSV, XLS, or XLSX) from your selected file and uses the appropriate method.

\subsection{Prerequisites}

\begin{itemize}
    \item Completed questionnaires must have ``\_Extract Data.csv'' files
    \item CSV files must be in the participant folder
    \item A masterfile must exist (.csv, .xls, or .xlsx format)
    \item The masterfile will be updated with new participant data
\end{itemize}

\subsection{Step-by-Step Guide}

\subsubsection{Step 1: Select Source Folder}
\begin{enumerate}
    \item Click \textbf{Search} next to ``Source folder PATH''
    \item Choose the folder containing participant folders
    \item Example: \texttt{C:\textbackslash Studies\textbackslash WaS3\textbackslash Completed}
\end{enumerate}

\subsubsection{Step 2: Select Masterfile}
\begin{enumerate}
    \item Click \textbf{Search} next to ``Masterfile PATH''
    \item Choose your masterfile (.csv, .xls, or .xlsx)
    \item The system will auto-detect the format and use the appropriate extraction method
\end{enumerate}

\subsubsection{Step 3: Enter Participant ID}
\begin{itemize}
    \item Enter the ID of the participant whose data you want to extract
    \item Must match the folder name exactly
    \item Case-sensitive
\end{itemize}

\subsubsection{Step 4: Extract Data}

Click \textbf{Extract Data to Masterfile}. The system will automatically detect your file format and proceed accordingly:

\paragraph{For CSV files (.csv)}
\begin{enumerate}
    \item Find all \texttt{*\_Extract Data.csv} files in the participant folder
    \item Read data from each CSV file
    \item Append a new row to the CSV masterfile
    \item Merge column headers if new fields appear
    \item Save the updated file
\end{enumerate}

\paragraph{For Excel files (.xls or .xlsx)}
\begin{enumerate}
    \item Find all \texttt{*\_Extract Data.csv} files in the participant folder
    \item Read data from each CSV file
    \item Open the Excel masterfile
    \item Find the next empty row
    \item Write participant ID and all extracted data
    \item Create column headers if this is the first row
    \item Save the file (XLS files stay as .xls, XLSX stay as .xlsx)
\end{enumerate}

\subsection{Choosing Masterfile Format}

\textbf{Use CSV (.csv) when:}
\begin{itemize}
    \item You want a universal, plain-text format
    \item You plan to import data into R, Python, SPSS, or other analysis tools
    \item You need maximum compatibility across platforms and software
    \item You want human-readable, version-control-friendly files
    \item File size is a concern (CSV is smaller)
\end{itemize}

\textbf{Use Excel (.xls or .xlsx) when:}
\begin{itemize}
    \item You want to work directly in Excel with formulas/pivot tables
    \item You need Excel-specific features (formatting, charts, multiple sheets)
    \item Your workflow is primarily in Microsoft Excel
    \item You're using older Excel versions (use .xls for Excel 2003 and earlier)
    \item You're using newer Excel versions (use .xlsx for Excel 2007+)
\end{itemize}

\textbf{All formats:}
\begin{itemize}
    \item Append new participants as rows to the masterfile
    \item Automatically merge column headers when new fields appear
    \item Preserve all previous data in the masterfile
    \item The tool auto-detects the format -- just select your file
\end{itemize}

\subsection{Data Organization}

\begin{itemize}
    \item Participant ID is written in the first column
    \item Each CSV field gets a column named \texttt{\{SurveyType\}\_\{FieldName\}}
    \item Example columns: \texttt{demographics\_Age}, \texttt{nasa1\_Mental\_Demand}, \texttt{nasa1\_Physical\_Demand}
    \item Data is sorted alphabetically by column name
    \item The ``File'' column from CSVs is automatically excluded
\end{itemize}

\subsubsection{Step 5: Verify Results}
\begin{itemize}
    \item A success message shows how many CSV files were processed
    \item Check the Excel file to verify data was written correctly
\end{itemize}

\section{Configuration Management}

Save and load your questionnaire configurations for repeated use.

\subsection{Save Configuration}

\begin{enumerate}
    \item Set up your questionnaires in the Generate section
    \item Go to \textbf{File $\rightarrow$ Save Configuration}
    \item Enter a descriptive configuration name
    \item Click \textbf{Save configuration}
\end{enumerate}

\subsubsection{What's Saved}

\begin{itemize}
    \item Target folder path
    \item Number of questionnaire types
    \item All questionnaire names, template paths, and copy counts
    \item Source folder path (for extraction)
    \item Excel file path (for extraction)
\end{itemize}

\subsubsection{Overwriting Configurations}

If a configuration with the same name exists, you'll be asked to confirm:
\begin{itemize}
    \item Choose \textbf{Yes} to replace the old configuration
    \item Choose \textbf{No} to cancel
\end{itemize}

\subsection{Load Configuration}

\begin{enumerate}
    \item Go to \textbf{File $\rightarrow$ Load/Delete Configuration}
    \item Select a configuration from the dropdown
    \item Click \textbf{Load configuration}
\end{enumerate}

All saved settings will be restored to the interface.

\subsection{Delete Configuration}

\begin{enumerate}
    \item Go to \textbf{File $\rightarrow$ Load/Delete Configuration}
    \item Select a configuration from the dropdown
    \item Click \textbf{Delete configuration}
    \item Confirm the deletion
\end{enumerate}

\section{Troubleshooting}

\subsection{Common Error Messages}

\subsubsection{``Please enter a participant ID!''}
\begin{itemize}
    \item Make sure you've entered an ID in the Participant ID field
    \item The field cannot be empty
\end{itemize}

\subsubsection{``Please select a target folder!''}
\begin{itemize}
    \item Click Search and choose a valid folder
    \item The folder must exist and be accessible
\end{itemize}

\subsubsection{``Please configure questionnaires!''}
\begin{itemize}
    \item Enter a number in ``Number of questionnaire types''
    \item Click Confirm to create configuration rows
\end{itemize}

\subsubsection{``Participant folder not found''}
\begin{itemize}
    \item Verify the Source folder is correct
    \item Check that the Participant ID matches the folder name exactly
    \item Folder names are case-sensitive
\end{itemize}

\subsubsection{``No Extract Data CSV files found''}
\begin{itemize}
    \item Check that questionnaires have been processed
    \item CSV files must end with ``\_Extract Data.csv''
    \item Verify files are in the participant folder
\end{itemize}

\subsubsection{``Folder already exists. Overwrite?''}
\begin{itemize}
    \item Choose \textbf{Yes} to replace existing files
    \item Choose \textbf{No} to cancel and preserve existing data
\end{itemize}

\subsection{Excel-Related Issues}

\begin{itemize}
    \item Ensure Excel is installed and accessible
    \item Close the Excel file if it's already open
    \item Check file permissions
    \item Verify the Excel file is not corrupted
\end{itemize}

\subsection{Best Practices}

\subsubsection{Use Consistent Naming}
\begin{itemize}
    \item Use the same participant ID format across all participants
    \item Example: WaS3\_001, WaS3\_002, WaS3\_003, etc.
    \item Avoid special characters in IDs
\end{itemize}

\subsubsection{Organize Templates}
\begin{itemize}
    \item Keep all questionnaire templates in one folder
    \item Use descriptive filenames
    \item Maintain backups of template files
\end{itemize}

\subsubsection{Save Configurations}
\begin{itemize}
    \item Create configurations for each study type
    \item Use descriptive configuration names
    \item Update configurations when templates change
\end{itemize}

\subsubsection{Backup Data}
\begin{itemize}
    \item Regularly backup your Excel data files
    \item Keep copies of completed questionnaires
    \item Use version control for configuration files
\end{itemize}

\subsubsection{Test First}
\begin{itemize}
    \item Test with a sample participant before processing real data
    \item Verify Excel output is formatted correctly
    \item Check all questionnaire templates load correctly
\end{itemize}

\section{Advanced Features}

\subsection{Multiple Questionnaire Copies}

When you need repeated measures:
\begin{enumerate}
    \item Set ``Number of copies'' to the desired amount
    \item Files will be numbered sequentially: \texttt{name1.pdf}, \texttt{name2.pdf}, etc.
    \item Useful for: NASA-TLX repeated measures, multiple trials, time-series data
\end{enumerate}

\subsection{Custom Excel Sheets}

The extraction looks for a sheet named ``Data'':
\begin{itemize}
    \item If found, data is written to this sheet
    \item If not found, data is written to the first sheet
    \item Headers are created automatically on the first row
    \item Existing headers are preserved and matched
\end{itemize}

\subsection{File Column Filtering}

The extraction automatically skips any column named ``File'' from CSV data to avoid redundant path information.

\section{Technical Information}

\subsection{Version History}

\subsubsection{v2.0 (February 2026)}
\begin{itemize}
    \item Dynamic questionnaire support (unlimited types)
    \item Individual configuration per questionnaire
    \item Improved data extraction algorithm
    \item Enhanced error handling
    \item Configuration save/load system
    \item Memory-optimized processing
    \item Comprehensive type hints
\end{itemize}

\subsubsection{v1.2 (Previous)}
\begin{itemize}
    \item Fixed questionnaire count
    \item Basic data extraction
    \item Simple configuration
\end{itemize}

\subsection{Credits}

\begin{tabular}{ll}
\textbf{Version:} & 2.0 \\
\textbf{Release Date:} & February 2026 \\
\end{tabular}

\subsection{Technology Stack}

\begin{itemize}
    \item Python 3.13.7
    \item PyQt5 5.15.11 (GUI framework)
    \item xlwings 0.33.20 (Excel integration)
    \item PyInstaller 6.14.2 (Executable packaging)
\end{itemize}

\section{Support and Contact}

For technical support or feature requests, please refer to your organization's support channels.

\vfill
\begin{center}
\rule{0.5\textwidth}{0.4pt}\\[0.5em]
\textit{Survey Workbench v2.0 -- User Manual}\\[0.5em]
\textit{February 2026}
\end{center}

\end{document}
